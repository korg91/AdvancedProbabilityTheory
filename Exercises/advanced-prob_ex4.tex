\documentclass[12pt,a4paper]{report}

\usepackage{amsmath}
\usepackage{bbm}
\usepackage[utf8]{inputenc}
\usepackage{longtable}
\usepackage{amsthm}
\usepackage{amscd}
\usepackage{amssymb}
\usepackage{amsfonts}
\usepackage{amsmath}
\usepackage{mathtools}
\usepackage{enumitem}
\usepackage[hyphens]{url}
\usepackage[scale=3]{ccicons}  % per le icone creative commons
\usepackage{hyperref}  % per i link nel pdf
\usepackage[rmargin=3.0cm,lmargin=3.0cm]{geometry}
%\usepackage{frontesp}  % prima pagina; il pacchetto frontesp.sty si trova nella stessa cartella del file .tex (deve essere adattato a mano)
\usepackage{setspace}  % per l'interlinea
\usepackage[english]{babel}  % per sillabazione
\usepackage[all]{xy} %diagrammi di funzioni
\usepackage{xspace} %per assicurare la corretta gestione degli spazi finali quando uso e.g. \AC. NB: sarebbe meglio trovare un'altra soluzione...cfr. http://tex.stackexchange.com/questions/15220/no-space-present-after-ensuremath



\theoremstyle{definition}
\newtheorem{teo}{Teorema}[section]  % resetta la numerazione dei teoremi per ogni capitolo
\newtheorem{defn}[teo]{Definizione}  % la numerazione delle definizioni dipende da quella dei teoremi
\newtheorem{es}[teo]{Esempio}  % idem
\newtheorem{oss}[teo]{Osservazione}  % idem
\newtheorem{prop}[teo]{Proposizione}  % idem
\newtheorem{lemma}[teo]{Lemma}  % idem
\newtheorem{corollario}[teo]{Corollario}  % idem

%%% inizio comandi per stile per teoremi: "numero. Titolo" %%%
\newtheoremstyle{num.custom-title}
  {\topsep}   % ABOVESPACE
  {\topsep}   % BELOWSPACE
  {\normalfont}  % BODYFONT
  {0pt}       % INDENT (empty value is the same as 0pt)
  {\bfseries} % HEADFONT
  {}         % HEADPUNCT
  {5pt plus 1pt minus 1pt} % HEADSPACE
  {\thmnumber{#2.}\thmnote{ #3}}
  
\theoremstyle{num.custom-title}  
\newtheorem{teo_custom-title}[teo]{} % per usarlo basta \begin{teo_custom-title}[<Titolo teorema>] (usa automaticamente la numerazione di [teo])
%%% fine comandi per stile per teoremi: "numero. Titolo" %%%

\newenvironment{claim}[1]{\par\noindent\underline{Claim:}\space#1}{} %per i claim
\newenvironment{claimproof}[1]{\par\noindent\underline{Proof:}\space#1}{\leavevmode\unskip\penalty9999 \hbox{}\nobreak\hfill\quad\hbox{$\blacksquare$}} %per le dimostrazioni dei claim


\DeclareMathOperator{\dom}{dom}
\DeclareMathOperator{\ran}{ran}
\DeclareMathOperator{\orb}{orb}
\DeclareMathOperator{\id}{id}
\DeclareMathOperator{\Zdv}{Zdv}
\DeclareMathOperator{\Hom}{Hom}
\DeclareMathOperator{\End}{End}
\DeclareMathOperator{\Ann}{Ann}
\DeclareMathOperator{\A}{\mathcal{A}}
\DeclareMathOperator{\B}{\mathcal{B}}
\DeclareMathOperator{\PP}{\mathcal{P}}
\DeclareMathOperator{\RR}{\mathcal{R}}
\DeclareMathOperator{\LL}{\mathcal{L}}
\DeclareMathOperator{\Hrtg}{\text{Hrtg}}
\DeclareMathOperator{\Ord}{\text{Ord}}
\DeclareMathOperator{\N}{\mathbb{N}}
\DeclareMathOperator{\R}{\mathbb{R}}
\DeclareMathOperator{\Z}{\mathbb{Z}}
\DeclareMathOperator{\E}{\mathbb{E}}
\DeclareMathOperator{\M}{\mathfrak{M}}
\DeclareMathOperator{\U}{\mathfrak{U}}
\DeclareMathOperator{\PPP}{\mathbb{P}}
\DeclareMathOperator{\a01}{\{0,1\}^{\star}}
\DeclareMathOperator{\imp}{\Rightarrow}
\DeclareMathOperator{\sm}{\setminus}
\DeclareMathOperator{\sse}{\subseteq}


\newcommand{\AC}{\ensuremath{\mathsf{AC}}\xspace}
\newcommand{\CC}{\ensuremath{\mathsf{CC}}\xspace}
\newcommand{\DC}{\ensuremath{\mathsf{DC}}\xspace}
\newcommand{\ZF}{\ensuremath{\mathsf{ZF}}\xspace}
\newcommand{\ZFC}{\ensuremath{\mathsf{ZFC}}\xspace}
\newcommand{\LS}{\ensuremath{\mathsf{LS}}\xspace}
\newcommand{\AMC}{\ensuremath{\mathsf{AMC}}\xspace}
\newcommand{\HRule}{\rule{\linewidth}{0.5mm}} %per la prima pagina
\newcommand{\veedot}{\mathbin{\mathaccent\cdot\vee}}

\renewcommand{\phi}{\varphi}
\renewcommand{\S}{\mathcal{S}}
\renewcommand{\P}{\mathbb{P}}
\renewcommand{\1}{\mathbbm{1}}
\renewcommand{\epsilon}{\varepsilon}
\renewcommand{\Im}{\operatorname{Im}}


%%%% INIZIO COMANDI PER EQUIVALENZE %%%%
\newcommand{\Implies}[2]{$\text{\ref{statement#1}}\!\implies\!\text{\ref{statement#2}}$}% X => Y
\newcommand{\punto}[1]{\item \label{statement#1}}


\newenvironment{equivalence}
    {\begin{enumerate}[label=(\arabic*),ref=(\arabic*)]
    }
    { 
	\end{enumerate}
    }
%%%% FINE COMANDI PER EQUIVALENZE %%%



% Interlinea 1.5
%\onehalfspacing  


%per le citazioni
\def\signed #1{{\leavevmode\unskip\nobreak\hfil\penalty50\hskip2em
  \hbox{}\nobreak\hfil(#1)%
  \parfillskip=0pt \finalhyphendemerits=0 \endgraf}}

\newsavebox\mybox
\newenvironment{aquote}[1]
  {\savebox\mybox{#1}\begin{quote}}
  {\signed{\usebox\mybox}\end{quote}}

%disabilita colore link
\hypersetup{%
    pdfborder = {0 0 0}
}

\begin{document}

\noindent Andrea Gadotti \hfill 22/11/2014

\paragraph{Exercise 22.} The $\sigma$-algebra of measurable sets of $(\Omega^*, \A^*, \P^*)=\left( \{0,1\}, \PP\{0,1\}, B(1,\frac{1}{2}) \right)^{\N}$ is generated by the set of the form 
\[
B=B_1 \times B_2 \times ... \times B_n \times \{0,1\} \times \{0,1\} \times ... 
\]
for some $n \in \N$ with $B_i \in \PP\{0,1\}$, i.e. $B_i=\emptyset, \{0\} ,\{1\}, \{0,1\}$. If $B_i=\emptyset$ for some $i$, then $B=\emptyset$ and its counterimage (the empty set) is of course measurable. Thus we can assume WLOG $B_i \neq \emptyset$ for all $i$. Let now $I \sse \{1,...,n\}$ be the set of indeces s.t. $i \in I \iff B_i=\{0,1\}$. In order to lighten the notation, suppose that $I=\{1,...,m\}$ with $m \leq n$. Then it is immediate to see that
\[
B= \biguplus_{\epsilon_1,...,\epsilon_m \in \{0,1\}} \{\epsilon_1\} \times \{\epsilon_2\} \times ... \times \{\epsilon_m\} \times B_{m+1} \times ... \times B_n \times \{0,1\} \times \{0,1\} \times ... 
\]
It is very easy to show that the preimage of every factor of the (disjoint) union is an interval, thus it is measurable, and therefore $B$ is measurable as well (the preimage of a union is the union of the preimages).\\
Thus the preimage of every set like $B$ is measurable, and hence every set of the $\sigma$-algebra generated by them is measurable as well. So $\tau$ is measurable.\\
To show that $\tau$ is injective, it is sufficient to observe that $\tau(\omega)$ is the binary expansion of $\omega$, and two different real numbers have different binary representations.\\
Observe that any $s \in \{0,1\}^{\N}$ is a binary representation of some $\omega \in [0,1)$, so for all $s \in \{0,1\}^{\N}$ there exists $\omega \in [0,1)$ s.t. $\tau(\omega)=s$, unless $\omega$ has multiple (two) possible representations. This happens if and only if $\omega$ is a dyadic rational, i.e. $\omega = \frac{a}{2^b}$ for $a \in \Z$ and $b \in \N$. Thus, the set $\Omega^* \sm \tau[\Omega]$ corresponds to the representations of dyadic rationals which are not given by $(X_n(\omega))_{n \in \N}$ (and looking at the definition of the $X_n$'s, they are the infinitely long ones). But since the dyadic rationals are of course countable, then also $\Omega^* \sm \tau[\Omega]$ is countable, and since every singleton of $\Omega^*$ has measure $0$ (easy to check, just take any sequence of sets decreasing to the singleton), we obtain $\P[\Omega^* \sm \tau[\Omega]]=0$ (since countable union of null-measure sets has measure $0$).\\
We want now to prove that $\P[\tau^{-1}A]=\P^*[A]$ for every $A \in \A^*$, i.e. $\P \circ \tau^{-1} \equiv \P^{*}$ are the same probability measure on $\Omega^*$. It is sufficient to check the equivalence on the rectangles (this is assured by many reasons: by definition of product measure, or by the fact that the rectangles are a semi-ring, or even a $\pi$-system). Consider then a rectangle $B$. Thanks to the first point (and using the same notation), the preimage of a rectangle is the disjoint union of $2^m$ intervals, and it is clear that all these intervals have measure $\frac{1}{2^n}$. Thus $\P \circ \tau^{-1} (B) = 2^m \cdot \frac{1}{2^n}=\frac{1}{2^{n-m}}$, which is exactly $\P^*[B]=\frac{1}{2^{m-n}}$ (since the only "fixed coordinates" are $m-n$).
%
% We want to check the equality only on the rectangles. We need the following:\\
%\textbf{Claim.} The family of the preimages of the rectangles forms a $\pi$-system of $[0,1)$ which is a subset of $\B$.
%\begin{proof}
%We have already proven that the preimage of a rectangle is a union of intervals (and then a borel set). Furthermore, the family of preimages is always closed under intersection, since $\tau^{-1}[B_1] \cap \tau^{-1}[B_2] = \tau^{-1}[B_1 \cap B_2]$.
%\end{proof}
%Thanks to the claim (and the $pi$-system theorem), we have that a probability measure on $([0,1),\B)$ is completely determined by its values on the preimages of the rectangles.

\paragraph{Exercise 24.} \ 
\\
\\
\textbf{$\pi$-system Lemma.} Let $(\Omega,\mathcal F, \P)$ be a probability space. Suppose that $\mathcal G$ and $\mathcal H$ are subsets of $\mathcal F$, and that $\mathcal I$ and $\mathcal J$ are $\pi$-systems with
\[
\sigma(\mathcal I) = \mathcal G, \ \sigma(\mathcal J)= \mathcal H.
\]
Then $\mathcal G$ and $\mathcal H$ are independent if and only if $\mathcal I$ and $\mathcal J$ are independent, that is
\[
\P[I \cap J] = \P[I] \P[J] \quad \text{ for all } I \in \mathcal I, J \in \mathcal J.
\]
\begin{proof}
We prove only the non-trivial implication. For any fixed $I \in \mathcal I$, the functions
\[
H \mapsto \P[I \cap H] \text{ and } H \mapsto \P[I]\P[H]
\]
are measures on $(\Omega, \mathcal H)$, they have the same total mass $\P[I]$ (and thus they are finite), and agree on $\mathcal J$. Therefore, by the second sheet of exercises\footnote{Actually, we know this result just for semirings and \emph{probability} measures. It is not a problem since: (1) The argument can be easily adapted to any couple of \emph{finite} measures with the same total mass (i.e. the total mass has not to be necessarily $1$). (2) We will apply this lemma in the case of semirings. Nevertheless, we aim to the maximum generality, therefore we prefer to talk about $\pi$-system, and we are allowed to do that because the (almost) uniqueness of measures also holds for simple $\pi$-systems.}, we know that they must agree on $\sigma(\mathcal J)=\mathcal H$. Hence, 
\[
\P[I \cap H] = \P[I] \P[H] \quad \text{ for all } I \in \mathcal I, H \in \mathcal H.
\]
Thus, for any fixed $H \in \mathcal H$, the functions
\[
G \mapsto \P[G \cap H] \text{ and } H \mapsto \P[G]\P[H]
\]
are measures on $(\Omega, \mathcal G)$, they have the same total mass $\P[H]$ (and thus they are finite), and agree on $\mathcal I$. They therefore agree on $\sigma(\mathcal I) = \mathcal G$. Hence,
\[
\P[G \cap H] = \P[G] \P[H] \quad \text{ for all } G \in \mathcal G, H \in \mathcal H
\]
and this is precisely what we wanted to prove.
\end{proof}
\begin{itemize}
 
\item[(a)] We follow the hint. \textbf{Note:} in order to lighten the notation, we deal with the particular case where the r.v. vectors are $(X_1,...,X_k)$ and $(X_{k+1},...,X_n)$, but it's clear that the argument can be repeated for every aribitrary disjoint couple of vectors $(X_{i_1},...,X_{i_k})$ and $(X_{j_1},...,X_{j_n})$.
\begin{claim} $\big\{[(X_1,...,X_k) \in B_1 \times ... \times B_k] \mid B_1,...,B_k \in \B(\R) \big\}$ is a $\pi$-system and $\A(X_1,...,X_k)=\sigma \left( \big\{[(X_1,...,X_k) \in B_1 \times ... \times B_k] \mid B_1,...,B_k \in \B(\R) \big\}\right)$.
\begin{claimproof}
Consider the event notation ``$[(X_1,...,X_k) \in B_1 \times ... \times B_k]$''. This is just another way to write $\gamma^{-1}[B_1 \times ... \times B_k]$, where $\gamma: \Omega \to \R^k$, $\gamma(\omega)= \big( X_1(\omega),...,X_k(\omega) \big)$. Thus, the fact that our collection is a $\pi$-system follows trivially by the fact that, for any function $f$, $f^{-1}(S_1 \cap S_2)=f^{-1}(S_1) \cap f^{-1}(S_2)$. But the last equality holds even for the intersection or the union of arbitrary many sets (and thus even for countably many). Therefore our family generates the whole $\sigma$-algebra $\A(X_1,...,X_k)$, since the rectangles generate the whole $\B(\R^k)$.
\end{claimproof}
\end{claim}

Thanks to the claim (and to the $\pi$-system lemma), we can check the independence only on (the preimages of) the rectangles. So
\begin{align*}
&\P \big[ [(X_1,...,X_k) \in B_1 \times ... \times B_k] \cap [(X_{k+1},...,X_n) \in B_{k+1} \times ... \times B_n] \big]\\
=& \P \big[ (X_1,...,X_k) \in B_1 \times ... \times B_k, (X_{k+1},...,X_n) \in B_{k+1} \times ... \times B_n \big]\\
=& \P \big[ (X_1 \in B_1,...,X_k \in B_k, X_{k+1} \in B_{k+1},...,X_n \in B_n \big]\\
= & \big( \P[X_1 \in B_1] \cdot ... \cdot \P[X_k \in B_k] \big) \cdot \big( \P[X_{k+1} \in B_{k+1}] \cdot ... \cdot \P[X_n \in B_n] \big) \\
=& \P \big[ X_1 \in B_1,...,X_k \in B_k \big] \cdot \P \big[X_{k+1} \in B_{k+1},...,X_n \in B_n \big]\\
=& \P \big[ (X_1,...,X_k) \in B_1 \times ... \times B_k \big] \cdot \P \big[ (X_{k+1},...,X_n) \in B_{k+1} \times ... \times B_n \big],
\end{align*}
as wanted. 
\item[(b)] Observe that
\[
\A(X_{k+1},X_{k+2},...) = \sigma \left( \bigcup_{n \in \N} \A(X_{k+1},...,X_{k+1+n}) \right)
\]
and that $\bigcup_{n \in \N} \A(X_{k+1},...,X_{k+1+n})$ is a $\pi$-system (trivial since the sequence of families is increasing, and thus every couple of sets is included in some element of the sequence, which of course contains their intersection).\\
So we can check the independence just between $\A(X_1,...,X_{k})$ and $\bigcup_{n \in \N} \A(X_{k+1},..., X_{k+1+n})$. But this follows trivially from point (a), since every element of $\bigcup_{n \in \N} \A(X_{k+1},...,X_{k+1+n})$ is an element of $\A(X_{k+1},...,X_{k+1+\overline{n}})$ for some $\overline{n} \in \N$.\\
We want now to show that $Y=f(X_1,...,X_k)$ and $Z=g(X_{k+1},X_{k+2},...)$ are indepentent. Let $B_1, B_2 \in \B(\R)$. Observe that
\begin{multline*}
\P[Y \in B_1, Z \in B_2]= \P\big[Y^{-1}[B_1] \cap Z^{-1}[B_2]\big]= \P\big[f^{-1}[B_1] \cap g^{-1}[B_2]\big]\\ 
=\P\big[(X_1,...,X_k) \in f^{-1}[B_1], (X_{k+1},X_{k+2},...) \in g^{-1}[B_2]\big].
\end{multline*}
But since $f$ and $g$ are measurable, we have that $f^{-1}[B_1]$ and $g^{-1}[B_2]$ are Borel sets. Therefore, thanks to point (b), we can split the last term into the product of the two probabilities, and we are done.
\item[(c)] It is sufficient to use (b) and repeat the same argument of the first part of (b): thanks to (b), $\A(Y_1,Y_2,...)$ is independent by $\A(X_1,...,X_n)$ for every $n \in \N$, and thus is independent by $\bigcup_{n \in \N} \A(X_1,...,X_n)$ and therefore is independent by $\sigma \left( \bigcup_{n \in \N} \A(X_1,...,X_n) \right) = \A(X_1,X_2,...)$.
\end{itemize}

\paragraph{Exercise 25.}
\begin{itemize}
\item[(a)] $\E \left( e^{itX} \right) = \theta e^{it} + (1-\theta)$.
\item[(b)] Since $X$ and $Y$ are independent, even $e^{itX}$ and $e^{itY}$ are independent, thus $\E \left( e^{it(X+Y)} \right) = \E \left( e^{itX}e^{itY} \right) = \E \left( e^{itX} \right) \E \left( e^{itY} \right) = \phi_X(t) \phi_Y(t)$.
\item[(c)] $X= Y_1 + ... + Y_n$ where $Y_n \sim B(1,\theta)$. Therefore $\phi_X(t) = \phi_{Y_1}(t) \cdot ... \cdot \phi_{Y_n}(t) = \left( \phi_{Y_1}(t) \right)^n$.
\end{itemize}

\paragraph{Exercise 26.}
\begin{itemize}
\item[(a)] $\phi_X(t)=\E[e^{itX}]=\sum_{x=0}^\infty [e^{it}]^x e^{-\lambda}\frac1{x!}\lambda^x=e^{-\lambda}\sum_{x=0}^{\infty}\frac{(\lambda e^{it})^x}{x!}=e^{-\lambda}e^{\lambda e^{it}}=e^{\lambda (e^{it}-1)}$.
\item[(b)] $\phi_X(t)=\E[e^{itX}]=\E[\cos(tX)+i\sin(tX)]=\E[\cos(tX)]+\E[i\sin(tX)]= \int_{0}^{1} \cos(tx) dx + i\int_{0}^{1} \sin(tx) dx = \frac{1}{t} (\sin(t)-\sin(0)) - \frac{i}{t} (\cos(t)-\cos(0))= \frac{\sin(t)}{t}-i \frac{\cos(t)-1}{t}$.
\item[(c)] $\phi_X(t)=\E[e^{itX}]= \int_0^{\infty} e^{itx}  \lambda e^{-\lambda x } dx = \lambda \int_0^{\infty} e^{(it-\lambda)x} dx = \frac{\lambda}{it - \lambda} e^{(it - \lambda)x}\bigg|_0^{\infty} = \frac{\lambda}{ \lambda- it}$. \\
Please note that, while the result is correct, this solution is actually wrong/highly informal, since we are basically using the substitution $z:=(it-\lambda)x$, so $z \in \mathbb C$, while $x \in \R$, and our integration theory for substitution doesn't cover this case. One possible solution is the boring way, i.e. to write $e^{itx}$ as $\cos(tx)+i\sin(tx)$ and do the computations (see \url{http://www.statlect.com/ucdexp1.htm}). \\
A more elegant approach relies on Complex Analysis results, and proceeds (euristically) this way: define $\gamma: \R^+ \to \mathbb C$ by $\gamma(x)=-\lambda x + it x$. Observe that it is a semi-diagonal in the fourth quadrant. Define $\delta: \R^+ \to \mathbb C$ by $\delta(x)=x$. If we can prove that $\int_\gamma e^z dz = \int_\delta e^z dz \ (=\int_{\R^+} e^{(it-\lambda)x} dx)$ we are done. Consider now the obvious right-angled triangles we can form with the restrictions of $\gamma$ and $\delta$. Since $e^z$ is olomorph, thanks to Cauchy's theorem we have that its integral on every such circuit is $0$. But the integral on the full triangle is the sum of the integrals computed on every single side of the triangle. But the integral on the vertical side goes to $0$ (or, at least, should do) when the length goes to infinity. So considering the limit to infinity of the sum we are done (since one of the two integrals must have ``minus'' sign because of ``orientation of the circuit'' reasons).
\end{itemize}





\end{document}