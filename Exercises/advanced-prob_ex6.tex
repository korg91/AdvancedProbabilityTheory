\documentclass[12pt,a4paper]{report}

\usepackage{amsmath}
\usepackage{bbm}
\usepackage[utf8]{inputenc}
\usepackage{longtable}
\usepackage{amsthm}
\usepackage{amscd}
\usepackage{amssymb}
\usepackage{amsfonts}
\usepackage{amsmath}
\usepackage{mathtools}
\usepackage{enumitem}
\usepackage[hyphens]{url}
\usepackage[scale=3]{ccicons}  % per le icone creative commons
\usepackage{hyperref}  % per i link nel pdf
\usepackage[rmargin=3.0cm,lmargin=3.0cm]{geometry}
%\usepackage{frontesp}  % prima pagina; il pacchetto frontesp.sty si trova nella stessa cartella del file .tex (deve essere adattato a mano)
\usepackage{setspace}  % per l'interlinea
\usepackage[english]{babel}  % per sillabazione
\usepackage[all]{xy} %diagrammi di funzioni
\usepackage{xspace} %per assicurare la corretta gestione degli spazi finali quando uso e.g. \AC. NB: sarebbe meglio trovare un'altra soluzione...cfr. http://tex.stackexchange.com/questions/15220/no-space-present-after-ensuremath



\theoremstyle{definition}
\newtheorem{teo}{Teorema}[section]  % resetta la numerazione dei teoremi per ogni capitolo
\newtheorem{defn}[teo]{Definizione}  % la numerazione delle definizioni dipende da quella dei teoremi
\newtheorem{es}[teo]{Esempio}  % idem
\newtheorem{oss}[teo]{Osservazione}  % idem
\newtheorem{prop}[teo]{Proposizione}  % idem
\newtheorem{lemma}[teo]{Lemma}  % idem
\newtheorem{corollario}[teo]{Corollario}  % idem

%%% inizio comandi per stile per teoremi: "numero. Titolo" %%%
\newtheoremstyle{num.custom-title}
  {\topsep}   % ABOVESPACE
  {\topsep}   % BELOWSPACE
  {\normalfont}  % BODYFONT
  {0pt}       % INDENT (empty value is the same as 0pt)
  {\bfseries} % HEADFONT
  {}         % HEADPUNCT
  {5pt plus 1pt minus 1pt} % HEADSPACE
  {\thmnumber{#2.}\thmnote{ #3}}
  
\theoremstyle{num.custom-title}  
\newtheorem{teo_custom-title}[teo]{} % per usarlo basta \begin{teo_custom-title}[<Titolo teorema>] (usa automaticamente la numerazione di [teo])
%%% fine comandi per stile per teoremi: "numero. Titolo" %%%

\newenvironment{claim}[1]{\par\noindent\underline{Claim:}\space#1}{} %per i claim
\newenvironment{claimproof}[1]{\par\noindent\underline{Proof:}\space#1}{\leavevmode\unskip\penalty9999 \hbox{}\nobreak\hfill\quad\hbox{$\blacksquare$}} %per le dimostrazioni dei claim


\DeclareMathOperator{\dom}{dom}
\DeclareMathOperator{\ran}{ran}
\DeclareMathOperator{\orb}{orb}
\DeclareMathOperator{\id}{id}
\DeclareMathOperator{\Zdv}{Zdv}
\DeclareMathOperator{\Hom}{Hom}
\DeclareMathOperator{\End}{End}
\DeclareMathOperator{\Ann}{Ann}
\DeclareMathOperator{\A}{\mathcal{A}}
\DeclareMathOperator{\B}{\mathcal{B}}
\DeclareMathOperator{\PP}{\mathcal{P}}
\DeclareMathOperator{\RR}{\mathcal{R}}
\DeclareMathOperator{\LL}{\mathcal{L}}
\DeclareMathOperator{\Hrtg}{\text{Hrtg}}
\DeclareMathOperator{\Ord}{\text{Ord}}
\DeclareMathOperator{\N}{\mathbb{N}}
\DeclareMathOperator{\Z}{\mathbb{Z}}
\DeclareMathOperator{\E}{\mathbb{E}}
\DeclareMathOperator{\M}{\mathfrak{M}}
\DeclareMathOperator{\U}{\mathfrak{U}}
\DeclareMathOperator{\Var}{Var}
\DeclareMathOperator{\PPP}{\mathbb{P}}
\DeclareMathOperator{\a01}{\{0,1\}^{\star}}
\DeclareMathOperator{\imp}{\Rightarrow}
\DeclareMathOperator{\sm}{\setminus}
\DeclareMathOperator{\sse}{\subseteq}

\newcommand{\R}{\mathbb{R}}


\newcommand{\AC}{\ensuremath{\mathsf{AC}}\xspace}
\newcommand{\CC}{\ensuremath{\mathsf{CC}}\xspace}
\newcommand{\DC}{\ensuremath{\mathsf{DC}}\xspace}
\newcommand{\ZF}{\ensuremath{\mathsf{ZF}}\xspace}
\newcommand{\ZFC}{\ensuremath{\mathsf{ZFC}}\xspace}
\newcommand{\LS}{\ensuremath{\mathsf{LS}}\xspace}
\newcommand{\AMC}{\ensuremath{\mathsf{AMC}}\xspace}
\newcommand{\HRule}{\rule{\linewidth}{0.5mm}} %per la prima pagina
\newcommand{\veedot}{\mathbin{\mathaccent\cdot\vee}}
\newcommand{\F}{\mathcal{F}}

\renewcommand{\phi}{\varphi}
\renewcommand{\S}{\mathcal{S}}
\renewcommand{\P}{\mathbb{P}}
\renewcommand{\1}{\mathbbm{1}}
\renewcommand{\epsilon}{\varepsilon}
\renewcommand{\Im}{\operatorname{Im}}


%%%% INIZIO COMANDI PER EQUIVALENZE %%%%
\newcommand{\Implies}[2]{$\text{\ref{statement#1}}\!\implies\!\text{\ref{statement#2}}$}% X => Y
\newcommand{\punto}[1]{\item \label{statement#1}}


\newenvironment{equivalence}
    {\begin{enumerate}[label=(\arabic*),ref=(\arabic*)]
    }
    { 
	\end{enumerate}
    }
%%%% FINE COMANDI PER EQUIVALENZE %%%



% Interlinea 1.5
%\onehalfspacing  


%per le citazioni
\def\signed #1{{\leavevmode\unskip\nobreak\hfil\penalty50\hskip2em
  \hbox{}\nobreak\hfil(#1)%
  \parfillskip=0pt \finalhyphendemerits=0 \endgraf}}

\newsavebox\mybox
\newenvironment{aquote}[1]
  {\savebox\mybox{#1}\begin{quote}}
  {\signed{\usebox\mybox}\end{quote}}

%disabilita colore link
\hypersetup{%
    pdfborder = {0 0 0}
}

\begin{document}

\noindent Andrea Gadotti \hfill 13/01/2015

\paragraph{Exercise 31.} Let $Y=\E[X \mid \F]$. We know that $Y$ is constant over every $C_j$, i.e. $Y_{|_{C_j}} = y_j \in \R$. Furthermore, we know
\[
y_j = \frac{\E[X \cdot \1_{C_j}]}{\P[C_j]}.
\]
Now observe that $X_{|_{C_j}}= -1$ for all $j \geq 3$. So $\E[X \cdot \1_{C_j}] = -1$. Since $\P[C_j] = 2^{-j+1}-2^{-j} = 2^{-j}$, we obtain
\[
y_j = - 2^{-j} \quad \forall j \geq 3.
\]
We shall compute $y_1$ and $y_2$. We have $C_1 = (1/2,1]$ and $C_2 = (1/4,1/2]$. So
\[
\E[X \cdot \1_{C_1}] = 0 \cdot \left( \frac{2}{3}-\frac{1}{2} \right) + 1 \cdot \left( 1-\frac{2}{3} \right) = \frac{1}{3}
\]
and
\[
\E[X \cdot \1_{C_2}] = -1 \cdot \left( \frac{1}{4}-0 \right) + 0 \cdot \left( \frac{1}{3}-\frac{1}{4} \right) = -\frac{1}{4}.
\]
Finally:
\[
y_1 = \frac{1}{3} \cdot 2^1 = \frac{2}{3} \quad \text{and} \quad y_2 = -\frac{1}{4} \cdot 2^2 = -1.
\]
\qed

\paragraph{Exercise 32.} Observe that, for every r.v. $X$, we have
\[
X = \frac{X(\omega)+X(-\omega)}{2} + \frac{X(\omega)-X(-\omega)}{2}.
\]
Now define $X_1(\omega) := \frac{X(\omega)+X(-\omega)}{2}$ and $X_2 := \frac{X(\omega)-X(-\omega)}{2}$. By linearity of conditional expectation, we get
$\E[X \mid \F] = \E[X_1 \mid \F] + \E[X_2 \mid \F]$.\\
Now let $A \in \F$. We have
\[
\int_A X_2 \ d\P = \frac{1}{2} \int_A X_2 \ d\lambda = 0,
\]
since $X_2$ is an odd function and $A$ is symmetric. So (by uniqueness) it follows that $\E[X_2 \mid \F] \equiv 0$.\\
Now observe that $X_1$ is an even function, and it is immediate to see that every even function is $\F$-measurable.
Thus, by the theory (but actually it's immediate by the definition of conditional expectation), $\E[X_1 \mid \F] = X$, whereby $\E[X \mid \F] = X$. \qed

\paragraph{Exercise 33.}
\newcommand{\Y}{\mathcal{Y}}

First, observe that $\F_{n-1} := \sigma(Y_1,...,Y_{n-1})$ is atomic. In fact, calling $\mathbf{Y} := (Y_1,...,Y_{n-1})$ the r.v. $\Omega \to \Y^{n-1}$, we have $\F_{n-1} = \sigma(\mathbf{Y}) = \{\mathbf{Y}^{-1}(S) \mid S \sse \Y^{n-1} \}$. 
Observe that
\[
\mathbf{Y}^{-1}(S) = \bigcup_{s \in S} \mathbf{Y}^{-1}(\{s\}).
\]
Since $\Y$ is countable, $\Y^{n-1}$ is countable as well, and so is every subset $S \sse \Y^{n-1}$. This implies that the union above is actually a countable union. It follows that
\[
\sigma(\mathbf{Y}) = \sigma \big( \{\mathbf{Y}^{-1}(\{s\}) \mid s \in \Y^{n-1} \} \big).
\] 
We have trivially 
\[
\biguplus_{s \in \Y^{n-1}} \mathbf{Y}^{-1}(\{s\}) = \Omega.
\]
Finally, since $\Y^{n-1}$ is countable, the union above is countable. So $\F_{n-1}$ is atomic, where the atoms are the sets of the form $\mathbf{Y}^{-1}(\{s\})$.\\
Now we can proceed with the exercise.
\begin{enumerate}
\item By the theory, we immediately have
\[
\E[X \mid Y_1, \ldots, Y_{n-1}] \upharpoonright \mathbf{Y}^{-1}(\{s\}) =
\frac{\E[X \cdot \1_{\mathbf{Y}^{-1}(\{s\})}]}{\P[\mathbf{Y}^{-1}(\{s\})]}
= \frac{\E[X \cdot \1_{\mathbf{Y}^{-1}(\{s\})}]}{p_{n-1}(s_1,\ldots,s_{n-1})}
\]
where $s = (s_1,...,s_{n-1})$.
\item If $X_n = g_n(Y_1,...,Y_n)$, then $X_n \cdot \1_{\mathbf{Y}^{-1}(\{s\})} = g_n(s_1,...,s_{n-1}, Y_n)$. So
\[
\E[X_n \mid Y_1, \ldots, Y_{n-1}] \upharpoonright \mathbf{Y}^{-1}(\{s\}) =
\frac{\E[g_n(s_1,\ldots,s_{n-1}, Y_n)]}{p_{n-1}(s_1,\ldots,s_{n-1})}.
\]
\item 
\begin{itemize}
\item $X_n$ is $\F_n$-measurable. By the way $X_n$ is defined, this is equivalent to say that $g_n$ is $\F_n$-measurable.
\item $\E[|X_n|] < \infty$ for all $n \in \N$.
\item $\E[X_n \mid \F_{n-1}] = X_{n-1}$, i.e. 
\[
\frac{\E[g_n(s_1,\ldots,s_{n-1}, Y_n)]}{p_{n-1}(s_1,\ldots,s_{n-1})} = X_{n-1} \upharpoonright \mathbf{Y}^{-1}(\{s\}), \quad \forall s \in \Y^{n-1}.
\]
\end{itemize}
\end{enumerate}


\end{document}