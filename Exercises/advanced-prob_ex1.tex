\documentclass[12pt,a4paper]{report}

\usepackage{amsmath}
\usepackage{bbm}
\usepackage[utf8]{inputenc}
\usepackage[italian]{babel}
\usepackage{longtable}
\usepackage{amsthm}
\usepackage{amscd}
\usepackage{amssymb}
\usepackage{amsfonts}
\usepackage{amsmath}
\usepackage{mathtools}
\usepackage{enumitem}
\usepackage[hyphens]{url}
\usepackage[scale=3]{ccicons}  % per le icone creative commons
\usepackage{hyperref}  % per i link nel pdf
\usepackage[rmargin=3.0cm,lmargin=3.0cm]{geometry}
%\usepackage{frontesp}  % prima pagina; il pacchetto frontesp.sty si trova nella stessa cartella del file .tex (deve essere adattato a mano)
\usepackage{setspace}  % per l'interlinea
\usepackage[italian]{babel}  % per sillabazione
\usepackage[all]{xy} %diagrammi di funzioni
\usepackage{xspace} %per assicurare la corretta gestione degli spazi finali quando uso e.g. \AC. NB: sarebbe meglio trovare un'altra soluzione...cfr. http://tex.stackexchange.com/questions/15220/no-space-present-after-ensuremath



\theoremstyle{definition}
\newtheorem{teo}{Teorema}[section]  % resetta la numerazione dei teoremi per ogni capitolo
\newtheorem{defn}[teo]{Definizione}  % la numerazione delle definizioni dipende da quella dei teoremi
\newtheorem{es}[teo]{Esempio}  % idem
\newtheorem{oss}[teo]{Osservazione}  % idem
\newtheorem{prop}[teo]{Proposizione}  % idem
\newtheorem{lemma}[teo]{Lemma}  % idem
\newtheorem{corollario}[teo]{Corollario}  % idem

%%% inizio comandi per stile per teoremi: "numero. Titolo" %%%
\newtheoremstyle{num.custom-title}
  {\topsep}   % ABOVESPACE
  {\topsep}   % BELOWSPACE
  {\normalfont}  % BODYFONT
  {0pt}       % INDENT (empty value is the same as 0pt)
  {\bfseries} % HEADFONT
  {}         % HEADPUNCT
  {5pt plus 1pt minus 1pt} % HEADSPACE
  {\thmnumber{#2.}\thmnote{ #3}}
  
\theoremstyle{num.custom-title}  
\newtheorem{teo_custom-title}[teo]{} % per usarlo basta \begin{teo_custom-title}[<Titolo teorema>] (usa automaticamente la numerazione di [teo])
%%% fine comandi per stile per teoremi: "numero. Titolo" %%%


\DeclareMathOperator{\dom}{dom}
\DeclareMathOperator{\ran}{ran}
\DeclareMathOperator{\orb}{orb}
\DeclareMathOperator{\id}{id}
\DeclareMathOperator{\Zdv}{Zdv}
\DeclareMathOperator{\Hom}{Hom}
\DeclareMathOperator{\End}{End}
\DeclareMathOperator{\Ann}{Ann}
\DeclareMathOperator{\A}{\mathcal{A}}
\DeclareMathOperator{\B}{\mathcal{B}}
\DeclareMathOperator{\PP}{\mathcal{P}}
\DeclareMathOperator{\LL}{\mathcal{L}}
\DeclareMathOperator{\Hrtg}{\text{Hrtg}}
\DeclareMathOperator{\Ord}{\text{Ord}}
\DeclareMathOperator{\N}{\mathbb{N}}
\DeclareMathOperator{\R}{\mathbb{R}}
\DeclareMathOperator{\Z}{\mathbb{Z}}
\DeclareMathOperator{\E}{\mathbb{E}}
\DeclareMathOperator{\M}{\mathfrak{M}}
\DeclareMathOperator{\U}{\mathfrak{U}}
\DeclareMathOperator{\PPP}{\mathbb{P}}
\DeclareMathOperator{\a01}{\{0,1\}^{\star}}
\DeclareMathOperator{\imp}{\Rightarrow}
\DeclareMathOperator{\sm}{\setminus}
\DeclareMathOperator{\sse}{\subseteq}


\newcommand{\AC}{\ensuremath{\mathsf{AC}}\xspace}
\newcommand{\CC}{\ensuremath{\mathsf{CC}}\xspace}
\newcommand{\DC}{\ensuremath{\mathsf{DC}}\xspace}
\newcommand{\ZF}{\ensuremath{\mathsf{ZF}}\xspace}
\newcommand{\ZFC}{\ensuremath{\mathsf{ZFC}}\xspace}
\newcommand{\LS}{\ensuremath{\mathsf{LS}}\xspace}
\newcommand{\AMC}{\ensuremath{\mathsf{AMC}}\xspace}
\newcommand{\HRule}{\rule{\linewidth}{0.5mm}} %per la prima pagina

\renewcommand{\phi}{\varphi}
\renewcommand{\S}{\mathcal{S}}
\renewcommand{\P}{\mathbb{P}}
\renewcommand{\1}{\mathbbm{1}}


%%%% INIZIO COMANDI PER EQUIVALENZE %%%%
\newcommand{\Implies}[2]{$\text{\ref{statement#1}}\!\implies\!\text{\ref{statement#2}}$}% X => Y
\newcommand{\punto}[1]{\item \label{statement#1}}


\newenvironment{equivalence}
    {\begin{enumerate}[label=(\arabic*),ref=(\arabic*)]
    }
    { 
	\end{enumerate}
    }
%%%% FINE COMANDI PER EQUIVALENZE %%%



% Interlinea 1.5
%\onehalfspacing  


%per le citazioni
\def\signed #1{{\leavevmode\unskip\nobreak\hfil\penalty50\hskip2em
  \hbox{}\nobreak\hfil(#1)%
  \parfillskip=0pt \finalhyphendemerits=0 \endgraf}}

\newsavebox\mybox
\newenvironment{aquote}[1]
  {\savebox\mybox{#1}\begin{quote}}
  {\signed{\usebox\mybox}\end{quote}}

%disabilita colore link
\hypersetup{%
    pdfborder = {0 0 0}
}

\begin{document}


\paragraph{Exercise 1.} $\P \left( \bigcup_{n=1}^\infty A_n \right) \le \sum_{n=1}^\infty \P(A_n)$
\begin{proof}
First of all, observe that any probability measure is monotone, since $B \supseteq A \imp \P(B)=\P(B \sm A \uplus A)=\P(B \sm A) + \P(A) \ge \P(A)$.\\
Define $A'_0:=A_0$ and $A'_n:=A_n \sm \bigcup_{i=1}^{n-1} A_i$. It's immediate to check that $\bigcup_{n=1}^\infty A'_n = \bigcup_{n=1}^\infty A_n$. Of course, $i \ne j \imp A'_i \cap A'_j \ne \emptyset$ and $A'_n \sse A_n$ for all $n$. Therefore
\[
\P \left( \bigcup_{n=1}^\infty A_n \right) = \P \left( \bigcup_{n=1}^\infty A'_n \right) = \sum_{n=1}^\infty \P(A'_n) \le \sum_{n=1}^\infty \P(A_n).
\]
\end{proof}

\paragraph{Exercise 2.}
TO-DO %$\liminf \P(A_n) = \sup\{\inf\{\P(A_m) \mid m \ge n\} \mid n \in \N\} \le  \P(\liminf A_n) = \P(\bigcup_{n=1}^\infty \bigcap_{m=n}^\infty A_m) \le \sum_{n=1}^\infty \P(\bigcap_{m=n}^\infty A_m)  $\\
$P(\liminf A_n)=\P(\bigcup_{n=1}^\infty \bigcap_{m=n}^\infty A_m)$. Define $B_n := \bigcap_{m=n}^\infty A_m$ and observe that $B_n \sse B_{n+1}$, thus by continuity $\P(\bigcup_{n=1}^\infty \bigcap_{m=n}^\infty A_m)=\P(\lim_{n \to \infty} \bigcup_{n=1}^\infty B_n) = \lim_{n \to \infty} \P(\bigcup_{n=1}^\infty B_n) = \liminf \P(\bigcup_{n=1}^\infty B_n)$

\paragraph{Exercise 3.} Prove the Borel-Cantelli lemma:
\[
\sum_{n=1}^\infty \P(A_n)<\infty \imp \P(\limsup A_n)=0.
\]
\begin{proof}
Since $\sum_{n=1}^\infty \P(A_n)<\infty$, we have necessarily that 
\[
\lim_{m \to \infty} \sum_{n=m}^\infty \P(A_n)=0. 
\]
Then, by monotony
\[
\P(\limsup A_n) = \P \left( \bigcap_{m=1}^\infty \bigcup_{n=m}^\infty A_n \right) \le \P \left( \bigcup_{n=m}^\infty A_n \right), \; \; \; \forall m \in \N
\]
But, by countable subadditivity
\[
\P \left( \bigcup_{n=m}^\infty A_n \right) \le \sum_{n=m}^\infty \P(A_n) \stackrel{n \to \infty}{\longrightarrow} 0
\]
and we are done.
\end{proof}

\paragraph{Exercise 4.} \
\begin{proof}
\begin{multline*}
\sum_{i \in I} \P(A \mid B_i) \P(B_i) = \sum_{\{i \in I : \P(B_i)\ne 0\}} \frac{\P(A \cap B_i)}{\P(B_i)} \P(B_i) = \sum_{i \in I} \P(A \cap B_i) = \\
 = \P \left( \biguplus_{i \in I} A \cap B_i \right) = \P \left( A \cap \biguplus_{i \in I} B_i \right) = \P(A).
\end{multline*}
\end{proof}

\paragraph{Exercise 5.} $\P(A \mid B)=\P(B \mid A) \cdot \kappa$. Determine $\kappa$.
\begin{proof}
Since $\P(A)=0 \vee \P(B)=0 \imp \P(A \mid B)=\P(B \mid A)=0$, in that case we can choose an arbitrary $\kappa$, for example $\kappa=0$.\\
If $\P(A) \ne 0 \ne \P(B)$, then $\P(B \mid A)=0 \imp \P(B \cap A)=0 \imp \P(A \mid B)=0$, and we can choose an arbitrary $\kappa$ again. If we suppose $P(A \mid B)$ too, then
\[
\frac{\P(A \mid B)}{\P(B \mid A)} = \frac{\ \ \frac{\P(A \cap B)}{\P(B)} \ \ }{\frac{\P(B \cap A)}{\P(A)}} = \frac{\P(A)}{\P(B)}.
\]
So $\kappa=\frac{\P(A)}{\P(B)}$.
\end{proof}

\paragraph{Exercise 6.}
\begin{itemize}
\item Let $D_i^C$ be the event ``the ball drawn from $i$-th urn is of colour $C$''. Of course $D_1^R \uplus D_1^W = \Omega$. So, by Exercise 4, we obtain
\[
\P(D_2^R) = \P(D_2^R \mid D_1^R) \cdot \P(D_1^R) + \P(D_2^R \mid D_1^W) \cdot \P(D_1^W) = 7/10 \cdot 3/10 + 6/10 \cdot 7/10 = 63/100.
\]
\item By Exercise 5 (Bayes' formula) we obtain
\[
\P(D_1^W \mid D_2^W) = \P(D_2^W \mid D_1^W) \frac{\P(D_1^W)}{\P(D_2^W)} = \P(D_2^W \mid D_1^W) \frac{\P(D_1^W)}{1-\P(D_2^R)} = 4/10 \cdot \frac{7/10}{37/100}=28/37.
\]
\end{itemize}

\paragraph{Exercise 7.} 
\begin{proof}
We want to prove that for any $n \in \N$ and any set $\{A_1,...,A_n\} \in \PP(\Omega)$ the following holds:
\begin{multline*}
\forall \{i_1,...,i_k\} \sse \{1,...,n\} \ \ \big[ \P(A_{i_1} \cap A_{i_2} \cap ... \cap A_{i_k}) = \P(A_{i_1}) \P(A_{i_2}) ... \P(A_{i_k}) \big] \\ 
\Updownarrow \\ 
\forall \{\epsilon_1,...,\epsilon_n\} \sse \{-1,1\} \ \ \big[ \P(A_1^{\epsilon_1} \cap ... \cap A_n^{\epsilon_n}) = \P(A_1^{\epsilon_1}) ... \P(A_n^{\epsilon_n}) \big]
\end{multline*}
$(\Longrightarrow)$ We proceed by induction on $N:=|\{\epsilon_i : \epsilon_i=-1\}|$. For the case $n=1$, observe that 
\[
\P(A)=\P(A \cap B) + \P(A \cap B^c),
\]
so
\[
\P(A \cap B^c)=\P(A)-\P(A \cap B)= \P(A)-\P(A) \P(B) =\P(A)(1-\P(B))=\P(A)\P(B^c)
\]
$(\Longleftarrow)$ Fix $n \in \N$. We proceed by induction on $d=n-k$.
If $d=0$ there is nothing to prove. Now suppose that the statement holds for $n-k=d$, i.e. for $k=n-d$. We want to show that it holds for $d=n-k+1$, i.e. for $n-d=k-1$. Observe that
\begin{multline*}
\P(A_{i_1} \cap ... \cap A_{i_{k-1}}) = \P(A_{i_1} \cap ... \cap A_{i_{k-1}} \cap B) + \P(A_{i_1} \cap ... \cap A_{i_{k-1}} \cap B^c) = \\
\P(A_{i_1}) ... \P(A_{i_{k-1}}) \P(B) + \P(A_{i_1}) ... \P(A_{i_{k-1}}) \P(B^c) = \\
\P(A_{i_1}) ... \P(A_{i_{k-1}}) \P(B) + \P(A_{i_1}) ... \P(A_{i_{k-1}}) (1-\P(B)) =  \P(A_{i_1}) ... \P(A_{i_{k-1}})
\end{multline*}
where the second equality holds thanks to the inductive hypothesis.\\
Since $n$ is arbitrary, the proof above holds for any set of events $\{A_1,...,A_n\}$.

%Suppose now that the following holds for any set $\{A_1,...,A_n\}$:
%\begin{multline*}
%\forall \{i_1,...,i_k\} \sse \{1,...,n\} \ \ \big[ \P(A_{i_1} \cap A_{i_2} \cap ... \cap A_{i_k}) = \P(A_{i_1}) \P(A_{i_2}) ... \P(A_{i_k}) \big] \\ 
%\Updownarrow \\ 
%\forall \{\epsilon_1,...,\epsilon_n\} \sse \{-1,1\} \ \ \big[ \P(A_1^{\epsilon_1} \cap ... \cap A_n^{\epsilon_n}) = \P(A_1^{\epsilon_1}) ... \P(A_n^{\epsilon_n}) \big]
%\end{multline*}
%We want to prove that the same holds for $n+1$.\\
%$(\Longrightarrow)$ Let $\{\epsilon_1,...,\epsilon_{n+1}\} \sse \{-1,1\}$ and suppose $\epsilon_{n+1}=-1$. Then
%\begin{multline*}
%\P(A_1^{\epsilon_1} \cap ... \cap A_n^{\epsilon_n})=\P(A_1^{\epsilon_1} \cap ... \cap A_n^{\epsilon_n} \cap A_{n+1}^{\epsilon_{n+1}}) + \P(A_1^{\epsilon_1} \cap ... \cap A_n^{\epsilon_n} \cap A_{n+1}^{\overline{\epsilon_{n+1}}})
%\end{multline*}
\end{proof}

\paragraph{Exercise 8.} 
\begin{proof}
$i \ne j \imp \P(A_i \cap A_j) = 1/4 = 2/4 \cdot 2/4 = \P(A_i) \P(A_j).$\\
But $\P(A_1 \cap A_2 \cap A_3)=1/4 \ne 2/4 \cdot 2/4 \cdot 2/4 = \P(A_1) \P(A_2) \P(A_3)$.
\end{proof}




\paragraph{Lezione 14/10}\ \\
\textbf{Remark:} d'ora in poi consideriamo le RV \emph{uguali} se $\P[X=X']=1$.

\begin{defn}
The \emph{expectation} (or \emph{expected value, mean}) of $X$ is 
\[
\E(X):=\int_{\Omega} X d\P
\]
if the integral exists.
\end{defn}

\textbf{Reminder:} Construction of Lebesgue integral:
\begin{enumerate}
\item Simple RV:
\[
X=\sum_{i=1}^n c_i \1
\]
with $c_i \in \R$, $A_i \in \A$. Then
\[
\E(X):=\sum_{i=1}^n c_i \P(A_i)
\]
\end{enumerate}

COPIARE


<PIC. I>











\end{document}