\documentclass[12pt,a4paper]{report}

\usepackage{amsmath}
\usepackage{bbm}
\usepackage[utf8]{inputenc}
\usepackage{longtable}
\usepackage{amsthm}
\usepackage{amscd}
\usepackage{amssymb}
\usepackage{amsfonts}
\usepackage{amsmath}
\usepackage{mathtools}
\usepackage{enumitem}
\usepackage[hyphens]{url}
\usepackage[scale=3]{ccicons}  % per le icone creative commons
\usepackage{hyperref}  % per i link nel pdf
\usepackage[rmargin=3.0cm,lmargin=3.0cm]{geometry}
%\usepackage{frontesp}  % prima pagina; il pacchetto frontesp.sty si trova nella stessa cartella del file .tex (deve essere adattato a mano)
\usepackage{setspace}  % per l'interlinea
\usepackage[english]{babel}  % per sillabazione
\usepackage[all]{xy} %diagrammi di funzioni
\usepackage{xspace} %per assicurare la corretta gestione degli spazi finali quando uso e.g. \AC. NB: sarebbe meglio trovare un'altra soluzione...cfr. http://tex.stackexchange.com/questions/15220/no-space-present-after-ensuremath



\theoremstyle{definition}
\newtheorem{teo}{Teorema}[section]  % resetta la numerazione dei teoremi per ogni capitolo
\newtheorem{defn}[teo]{Definizione}  % la numerazione delle definizioni dipende da quella dei teoremi
\newtheorem{es}[teo]{Esempio}  % idem
\newtheorem{oss}[teo]{Osservazione}  % idem
\newtheorem{prop}[teo]{Proposizione}  % idem
\newtheorem{lemma}[teo]{Lemma}  % idem
\newtheorem{corollario}[teo]{Corollario}  % idem

%%% inizio comandi per stile per teoremi: "numero. Titolo" %%%
\newtheoremstyle{num.custom-title}
  {\topsep}   % ABOVESPACE
  {\topsep}   % BELOWSPACE
  {\normalfont}  % BODYFONT
  {0pt}       % INDENT (empty value is the same as 0pt)
  {\bfseries} % HEADFONT
  {}         % HEADPUNCT
  {5pt plus 1pt minus 1pt} % HEADSPACE
  {\thmnumber{#2.}\thmnote{ #3}}
  
\theoremstyle{num.custom-title}  
\newtheorem{teo_custom-title}[teo]{} % per usarlo basta \begin{teo_custom-title}[<Titolo teorema>] (usa automaticamente la numerazione di [teo])
%%% fine comandi per stile per teoremi: "numero. Titolo" %%%


\DeclareMathOperator{\dom}{dom}
\DeclareMathOperator{\ran}{ran}
\DeclareMathOperator{\orb}{orb}
\DeclareMathOperator{\id}{id}
\DeclareMathOperator{\Zdv}{Zdv}
\DeclareMathOperator{\Hom}{Hom}
\DeclareMathOperator{\End}{End}
\DeclareMathOperator{\Ann}{Ann}
\DeclareMathOperator{\A}{\mathcal{A}}
\DeclareMathOperator{\B}{\mathcal{B}}
\DeclareMathOperator{\PP}{\mathcal{P}}
\DeclareMathOperator{\RR}{\mathcal{R}}
\DeclareMathOperator{\LL}{\mathcal{L}}
\DeclareMathOperator{\Hrtg}{\text{Hrtg}}
\DeclareMathOperator{\Ord}{\text{Ord}}
\DeclareMathOperator{\N}{\mathbb{N}}
\DeclareMathOperator{\R}{\mathbb{R}}
\DeclareMathOperator{\Z}{\mathbb{Z}}
\DeclareMathOperator{\E}{\mathbb{E}}
\DeclareMathOperator{\M}{\mathfrak{M}}
\DeclareMathOperator{\U}{\mathfrak{U}}
\DeclareMathOperator{\PPP}{\mathbb{P}}
\DeclareMathOperator{\a01}{\{0,1\}^{\star}}
\DeclareMathOperator{\imp}{\Rightarrow}
\DeclareMathOperator{\sm}{\setminus}
\DeclareMathOperator{\sse}{\subseteq}


\newcommand{\AC}{\ensuremath{\mathsf{AC}}\xspace}
\newcommand{\CC}{\ensuremath{\mathsf{CC}}\xspace}
\newcommand{\DC}{\ensuremath{\mathsf{DC}}\xspace}
\newcommand{\ZF}{\ensuremath{\mathsf{ZF}}\xspace}
\newcommand{\ZFC}{\ensuremath{\mathsf{ZFC}}\xspace}
\newcommand{\LS}{\ensuremath{\mathsf{LS}}\xspace}
\newcommand{\AMC}{\ensuremath{\mathsf{AMC}}\xspace}
\newcommand{\HRule}{\rule{\linewidth}{0.5mm}} %per la prima pagina

\renewcommand{\phi}{\varphi}
\renewcommand{\S}{\mathcal{S}}
\renewcommand{\P}{\mathbb{P}}
\renewcommand{\1}{\mathbbm{1}}
\renewcommand{\epsilon}{\varepsilon}


%%%% INIZIO COMANDI PER EQUIVALENZE %%%%
\newcommand{\Implies}[2]{$\text{\ref{statement#1}}\!\implies\!\text{\ref{statement#2}}$}% X => Y
\newcommand{\punto}[1]{\item \label{statement#1}}


\newenvironment{equivalence}
    {\begin{enumerate}[label=(\arabic*),ref=(\arabic*)]
    }
    { 
	\end{enumerate}
    }
%%%% FINE COMANDI PER EQUIVALENZE %%%



% Interlinea 1.5
%\onehalfspacing  


%per le citazioni
\def\signed #1{{\leavevmode\unskip\nobreak\hfil\penalty50\hskip2em
  \hbox{}\nobreak\hfil(#1)%
  \parfillskip=0pt \finalhyphendemerits=0 \endgraf}}

\newsavebox\mybox
\newenvironment{aquote}[1]
  {\savebox\mybox{#1}\begin{quote}}
  {\signed{\usebox\mybox}\end{quote}}

%disabilita colore link
\hypersetup{%
    pdfborder = {0 0 0}
}

\begin{document}

\paragraph{Exercise 10.}
\begin{proof}
To prove that the two sums are equal, first observe that 
\[
\{|X|>n\}= \biguplus_{k \geq n} \{k < |X| \leq k+1\}
\]
and thus 
\[
\P[|X|>n] = \sum_{k=n}^\infty \P[k < |X| \leq k+1].
\]
Therefore
\begin{multline*}
\sum_{n=1}^\infty \P[|X|>n] = \sum_{n=1}^\infty \sum_{k=n}^\infty \P[k < |X| \leq k+1] = \sum_{k=1}^\infty \sum_{n=1}^k \P[k < |X| \leq k+1] =\\
\sum_{k=1}^\infty \P[k < |X| \leq k+1] \sum_{n=1}^k 1 = \sum_{k=1}^\infty \P[k < |X| \leq k+1] \cdot k.
\end{multline*}
So the two sums are equal. Now we want to prove that 
\[
\int_\Omega |X| d\P < \infty \iff \sum_{k=1}^\infty k \P[k < |X| \leq k+1] < \infty.
\]
($\Longrightarrow$)
\begin{multline*}
\infty > \int_\Omega |X| d\P > \int_\Omega \sum_{k=1}^\infty \left( k \cdot \1_{\{\omega : k<|X(\omega)| \leq k+1\}} \right) d\P= \sum_{k=1}^\infty \int_\Omega k \cdot \1_{\{\omega : k<|X(\omega)| \leq k+1\}} d\P=\\
\sum_{k=1}^\infty k \int_\Omega \1_{\{\omega : k<|X(\omega)| \leq k+1\}} d\P = \sum_{k=1}^\infty k \P[k < |X| \leq k+1].
\end{multline*}
($\Longleftarrow$)
\begin{multline*}
\int_\Omega |X| d\P \leq \int_\Omega \sum_{k=0}^\infty \left( (k+1) \cdot \1_{\{\omega : k<|X(\omega)| \leq k+1\}} \right) d\P = \sum_{k=0}^\infty \int_\Omega (k+1) \cdot \1_{\{\omega : k<|X(\omega)| \leq k+1\}} d\P=\\
\sum_{k=0}^\infty (k+1) \int_\Omega \1_{\{\omega : k<|X(\omega)| \leq k+1\}} d\P = \sum_{k=0}^\infty (k+1) \P[k < |X| \leq k+1] < \infty.
\end{multline*}

\end{proof}











\paragraph{Exercise 11.} 
\begin{proof}
Suppose $\sum^{\infty}_{n=1}\P(A_n)=\infty$. We want to show $1-\P(\limsup_{n \rightarrow \infty} A_n) = 0$. First of all note that for all $N \in \N$ we have $\sum^{\infty}_{n=N}\P(A_n)=\infty$. Now observe that
\begin{multline*}
1 - \P(\limsup_{n \rightarrow \infty} A_n)  = 
1 - \P\left(\left(\bigcap_{N=1}^{\infty} \bigcup_{n=N}^{\infty}A_n\right)\right) =
\P\left(\left(\bigcap_{N=1}^{\infty} \bigcup_{n=N}^{\infty}A_n\right)^c \, \right) =\\
=\P\left(\bigcup_{N=1}^{\infty} \bigcap_{n=N}^{\infty}A_n^{c}\right)
= \P\left(\liminf_{n \rightarrow \infty}A_n^{c}\right)= \lim_{N \rightarrow \infty}\P\left(\bigcap_{n=N}^{\infty}A_n^{c}\right).
\end{multline*}
So it is enough to show that $\P\left(\bigcap_{n=N}^{\infty}A_n^{c}\right) = 0$ for all $N \in \N$. Since the $(A_n)^{\infty}_{n = 1}$ are independent and $1-x \leq e^{-x}$ for all $x \in \R^+$:
\begin{align*}
\P\left(\bigcap_{n=N}^{\infty}A_n^{c}\right) 
&= \prod^{\infty}_{n=N}\P\left(A_n^{c}\right) \\
&= \prod^{\infty}_{n=N}\left(1-\P\left(A_n\right)\right) \\
&\leq \prod^{\infty}_{n=N}e^{-\P(A_n)}\\
&=e^{-\sum^{\infty}_{n=N}\P(A_n)}\\
&=e^{-\infty}\\
&= 0,
\end{align*}
and we are done.\\
The first $(\Longrightarrow)$ is the Borel-Cantelli lemma (already proven). So we proved both the $(\Longrightarrow)$'s implications. Now it's just a matter of elementary logic to see that the $(\Longleftarrow)$'s hold as well (observe before that the sums must exist since they are sums of non-negative numbers).
\end{proof}

\paragraph{Exercise 12.} Let $\S$ be a semiring over $X$ and let $\RR(\S)$ the ring generated by $\S$. Then
\[
\RR(\S)=\left\{ A \sse X \mid A=\biguplus_{i=1}^n S_i \text{ for some disjoint elements of } \S \right\} =:\A
\]
\begin{proof}
The inclusion $\supseteq$ is trivial.\\
We will show now that $\A$ is a ring. Take any $A,B \in \A$, i.e.
\[
A=\bigcup_{i=1}^m S_i \text{ and } B=\bigcup_{j=1}^n T_j
\]
for some disjoint collections $\{S_i\}$ and $\{T_j\}$ in $\S$. Then
\begin{align*}
A \sm B
&= \left( \bigcup_{i=1}^m S_i \right) \cap \left( \bigcup_{j=1}^n T_j \right)^c\\
&= \left( \bigcup_{i=1}^m S_i \right) \cap \left( \bigcap_{j=1}^n T_j^c \right)\\
&= \bigcup_{i=1}^m \left(  S_i \cap \left( \bigcap_{j=1}^n T_j^c \right)\right)\\
&= \bigcup_{i=1}^m \left\{ \bigcap_{j=1}^n  (S_i \sm T_j) \right\}. \tag{a}
\end{align*}
Since $\S$ is a semiring, we have 
\[
S_i \sm T_j = \biguplus_{l=1}^{L_{ij}} H_{ijl}
\]
for some disjoint $\{H_{ijl}\}$ of $\S$. Thus
\begin{align*}
\bigcap_{j=1}^n  (S_i \sm T_j)
&=\bigcap_{j=1}^n \biguplus_{l=1}^{L_{ij}} H_{ijl}\\
&= \left( \biguplus_{l_1=1}^{L_{i1}} H_{i1l_1} \right) \cap \left( \biguplus_{l_2=1}^{L_{i2}} H_{i2l_2} \right) \cap ... \cap \left( \biguplus_{l_n=1}^{L_{in}} H_{inl_n} \right)\\
&= \bigcup_{l_1=1}^{L_{i1}} \bigcup_{l_2=1}^{L_{i2}} ...  \bigcup_{l_n=1}^{L_{in}} (H_{i1l_1} \cap H_{i2l_2} \cap ... \cap  H_{inl_n}), \tag{b}
\end{align*}
which is a disjoint union. Furthermore, since $\S$ is a semiring, we have
\[
H_{i1l_1} \cap H_{i2l_2} \cap ... \cap  H_{inl_n} \in \S.
\]
Therefore, combining (a) and (b), we see that $A \sm B$ is a disjoint union of sets of $\S$, i.e. $A \sm B \in \A$.\\
A similar argument shows that $S_i \cup T_j \in \S$ as well. So 
\[
A \cap B = \left( \bigcup_{i=1}^m S_i \right) \cap \left( \bigcup_{j=1}^n T_j \right) = \bigcup_{i=1}^m \bigcup_{j=1}^n (S_i \cap T_j) \in \A.
\]
Then $A \cup B = (A \sm B) \cup (B \sm A) \cup (A \cap B)$ is a disjoint union of sets that we have seen above are in $\A$, so $A \cup B \in \A$.\\
We have therefore proven that $\A$ is a ring. Since $\S \sse \A$, this implies $\RR(\S) \sse \A$.
\end{proof}
Let now $\mu : \S \to [0,+\infty)$ be a finitely additive (finite) measure on $\S$. Then $\mu^* : \RR(\S) \to [0,+\infty)$ given by
\[
\mu^*(A)=\mu^* \left( \biguplus_{i=1}^n S_i \right) = \sum_{i=1}^n \mu(S_i)
\]
is the unique extension of $\mu$ to a finitely additive measure on $\RR(\S)$.
\begin{proof}
The proof that $\mu^*$ is a measure on $\RR(\S)$ is immediate.\\
It is also trivial that it is unique, since 
\[
\mu'(A) \neq \mu^*(A) \imp 
\mu' \left( \biguplus_{i=1}^n S_i \right) \neq \mu^* \left( \biguplus_{i=1}^n S_i \right) \imp  \sum_{i=1}^n \mu'(S_i) \neq  \sum_{i=1}^n \mu(S_i)
\]
which means that $\mu'$ does not extend $\mu$.
\end{proof}

\paragraph{Exercise 13.}
\begin{proof}
First observe that $\mu^*$ is monotone:
\[
A \sse B \imp \mu^*(B) = \mu^*(A \uplus (B \sm A)) = \mu^*(A)+\mu^*(B \sm A) \geq \mu^*(A).
\]
Thus $\mu^*(\biguplus_{i=1}^\infty A_i) \geq \mu^*(\biguplus_{i=1}^N A_i)$ for all $N \in \N$ (note that a priori $\mu$ is not defined on the partial union). So we obtain
\begin{multline*}
\mu \left( \biguplus_{i=1}^\infty A_i \right) = \mu^* \left( \biguplus_{i=1}^\infty A_i \right) \geq \sup_{N \in \N} \left\{ \mu^* \left( \biguplus_{i=1}^N A_i \right) \right\}= \\
\sup_{N \in \N} \left\{\sum_{i=1}^N \mu^*(A_i) \right\} = \sum_{i=1}^\infty \mu^*(A_i) = \sum_{i=1}^\infty \mu(A_i).
\end{multline*}
\end{proof}

\paragraph{Exercise 14.}
\begin{proof}
Immediate.
\end{proof}


\paragraph{Exercise 15.}
\begin{proof}
First of all, observe that every open and closed interval is an element of $\RR(\S)$, since $\RR(\S)$ is closed under complement (and intersection).\\
Choose an arbitrary $\epsilon>0$. Since $F$ is right continuous, we can find $d>a$ such that $F(a) \leq F(d) \leq F(a) + \epsilon$. This means that 
\[
\mu((d,b])-\mu((a,b])=F(b)-F(d)-(F(b)-F(a))=F(a)-F(d) \geq - \epsilon,
\]
i.e. $\mu((d,b]) \geq \mu((a,b])- \epsilon$.

Thus, consider the closed interval $[d,b] \sse (a,b]$. We trivially obtain
\[
\mu^*([d,b]) \geq \mu((d,b]) \geq \mu((a,b]) - \epsilon.
\]
\textbf{Claim.} For all $n \in \N$ we can find $(a_n,d_n) \supseteq (a_n,b_n]$ s.t.
\[
\sum_{n=1}^\infty \mu^*((a_n,d_n)) \leq \sum_{n=1}^\infty \mu^*((a_n,b_n])+\epsilon.
\]
We will prove the claim later. Now observe that $\{(a_n,d_n)\}_{n \in \N}$ is an open cover of $[d,b]$, which is a compact set. Thus there exists a finite open subcover $\{(a_n,d_n)\}_{n \in F}$ of $[d,b]$. Therefore
\[
\mu^*([d,b]) \leq \mu^* \left( \bigcup_{n \in F} (a_n,d_n) \right) \leq \sum_{n \in F} \mu^*((a_n,d_n)) \leq \sum_{n=1}^\infty \mu^*((a_n,d_n)),
\]
where the second inequality holds because every additive measure is finitely sub-additive (to show this, just repeat the proof of the Exercise 1. in the finite case). Therefore
\[
\mu((a,b]) - \epsilon \leq \mu^*([d,b]) \leq \sum_{n=1}^\infty \mu^*((a_n,d_n)) \leq \sum_{n=1}^\infty \mu^*((a_n,b_n])+\epsilon.
\]
that is
\[
\mu((a,b]) \leq \sum_{n=1}^\infty \mu((a_n,b_n])+2\epsilon.
\]
Thanks to the arbitrary choice of $\epsilon$, we are done.
\end{proof}
\noindent\emph{Proof of the Claim.} Take any $b_n$. Since $F$ is right continuous, we can find $d_n > b_n$ such that $F(b_n) \leq F(d_n) \leq F(b_n) + \frac{\epsilon}{2^n}$. This means that
\[
\mu((a_n,b_n])-\mu((a_n,d_n])=F(b_n)-F(a_n)-(F(d_n)-F(a_n))=F(b_n)-F(d_n) \geq - \frac{\epsilon}{2^n},
\]
i.e. $\mu^*((a_n,d_n)) \leq \mu((a_n,d_n]) \leq \mu((a_n,b_n])+ \frac{\epsilon}{2^n}$.\\
Thus, we obtain
\[
\sum_{n=1}^\infty \mu^*((a_n,d_n)) \leq \sum_{n=1}^\infty \left( \mu^*((a_n,b_n])+\frac{\epsilon}{2^n} \right) = \sum_{n=1}^\infty \mu^*((a_n,b_n])+\epsilon,
\]
which is what we wanted to prove. \qed



\end{document}