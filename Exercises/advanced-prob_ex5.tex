\documentclass[12pt,a4paper]{report}

\usepackage{amsmath}
\usepackage{bbm}
\usepackage[utf8]{inputenc}
\usepackage{longtable}
\usepackage{amsthm}
\usepackage{amscd}
\usepackage{amssymb}
\usepackage{amsfonts}
\usepackage{amsmath}
\usepackage{mathtools}
\usepackage{enumitem}
\usepackage[hyphens]{url}
\usepackage[scale=3]{ccicons}  % per le icone creative commons
\usepackage{hyperref}  % per i link nel pdf
\usepackage[rmargin=3.0cm,lmargin=3.0cm]{geometry}
%\usepackage{frontesp}  % prima pagina; il pacchetto frontesp.sty si trova nella stessa cartella del file .tex (deve essere adattato a mano)
\usepackage{setspace}  % per l'interlinea
\usepackage[english]{babel}  % per sillabazione
\usepackage[all]{xy} %diagrammi di funzioni
\usepackage{xspace} %per assicurare la corretta gestione degli spazi finali quando uso e.g. \AC. NB: sarebbe meglio trovare un'altra soluzione...cfr. http://tex.stackexchange.com/questions/15220/no-space-present-after-ensuremath



\theoremstyle{definition}
\newtheorem{teo}{Teorema}[section]  % resetta la numerazione dei teoremi per ogni capitolo
\newtheorem{defn}[teo]{Definizione}  % la numerazione delle definizioni dipende da quella dei teoremi
\newtheorem{es}[teo]{Esempio}  % idem
\newtheorem{oss}[teo]{Osservazione}  % idem
\newtheorem{prop}[teo]{Proposizione}  % idem
\newtheorem{lemma}[teo]{Lemma}  % idem
\newtheorem{corollario}[teo]{Corollario}  % idem

%%% inizio comandi per stile per teoremi: "numero. Titolo" %%%
\newtheoremstyle{num.custom-title}
  {\topsep}   % ABOVESPACE
  {\topsep}   % BELOWSPACE
  {\normalfont}  % BODYFONT
  {0pt}       % INDENT (empty value is the same as 0pt)
  {\bfseries} % HEADFONT
  {}         % HEADPUNCT
  {5pt plus 1pt minus 1pt} % HEADSPACE
  {\thmnumber{#2.}\thmnote{ #3}}
  
\theoremstyle{num.custom-title}  
\newtheorem{teo_custom-title}[teo]{} % per usarlo basta \begin{teo_custom-title}[<Titolo teorema>] (usa automaticamente la numerazione di [teo])
%%% fine comandi per stile per teoremi: "numero. Titolo" %%%

\newenvironment{claim}[1]{\par\noindent\underline{Claim:}\space#1}{} %per i claim
\newenvironment{claimproof}[1]{\par\noindent\underline{Proof:}\space#1}{\leavevmode\unskip\penalty9999 \hbox{}\nobreak\hfill\quad\hbox{$\blacksquare$}} %per le dimostrazioni dei claim


\DeclareMathOperator{\dom}{dom}
\DeclareMathOperator{\ran}{ran}
\DeclareMathOperator{\orb}{orb}
\DeclareMathOperator{\id}{id}
\DeclareMathOperator{\Zdv}{Zdv}
\DeclareMathOperator{\Hom}{Hom}
\DeclareMathOperator{\End}{End}
\DeclareMathOperator{\Ann}{Ann}
\DeclareMathOperator{\A}{\mathcal{A}}
\DeclareMathOperator{\B}{\mathcal{B}}
\DeclareMathOperator{\PP}{\mathcal{P}}
\DeclareMathOperator{\RR}{\mathcal{R}}
\DeclareMathOperator{\LL}{\mathcal{L}}
\DeclareMathOperator{\Hrtg}{\text{Hrtg}}
\DeclareMathOperator{\Ord}{\text{Ord}}
\DeclareMathOperator{\N}{\mathbb{N}}
\DeclareMathOperator{\Z}{\mathbb{Z}}
\DeclareMathOperator{\E}{\mathbb{E}}
\DeclareMathOperator{\M}{\mathfrak{M}}
\DeclareMathOperator{\U}{\mathfrak{U}}
\DeclareMathOperator{\Var}{Var}
\DeclareMathOperator{\PPP}{\mathbb{P}}
\DeclareMathOperator{\a01}{\{0,1\}^{\star}}
\DeclareMathOperator{\imp}{\Rightarrow}
\DeclareMathOperator{\sm}{\setminus}
\DeclareMathOperator{\sse}{\subseteq}

\newcommand{\R}{\mathbb{R}}


\newcommand{\AC}{\ensuremath{\mathsf{AC}}\xspace}
\newcommand{\CC}{\ensuremath{\mathsf{CC}}\xspace}
\newcommand{\DC}{\ensuremath{\mathsf{DC}}\xspace}
\newcommand{\ZF}{\ensuremath{\mathsf{ZF}}\xspace}
\newcommand{\ZFC}{\ensuremath{\mathsf{ZFC}}\xspace}
\newcommand{\LS}{\ensuremath{\mathsf{LS}}\xspace}
\newcommand{\AMC}{\ensuremath{\mathsf{AMC}}\xspace}
\newcommand{\HRule}{\rule{\linewidth}{0.5mm}} %per la prima pagina
\newcommand{\veedot}{\mathbin{\mathaccent\cdot\vee}}

\renewcommand{\phi}{\varphi}
\renewcommand{\S}{\mathcal{S}}
\renewcommand{\P}{\mathbb{P}}
\renewcommand{\1}{\mathbbm{1}}
\renewcommand{\epsilon}{\varepsilon}
\renewcommand{\Im}{\operatorname{Im}}


%%%% INIZIO COMANDI PER EQUIVALENZE %%%%
\newcommand{\Implies}[2]{$\text{\ref{statement#1}}\!\implies\!\text{\ref{statement#2}}$}% X => Y
\newcommand{\punto}[1]{\item \label{statement#1}}


\newenvironment{equivalence}
    {\begin{enumerate}[label=(\arabic*),ref=(\arabic*)]
    }
    { 
	\end{enumerate}
    }
%%%% FINE COMANDI PER EQUIVALENZE %%%



% Interlinea 1.5
%\onehalfspacing  


%per le citazioni
\def\signed #1{{\leavevmode\unskip\nobreak\hfil\penalty50\hskip2em
  \hbox{}\nobreak\hfil(#1)%
  \parfillskip=0pt \finalhyphendemerits=0 \endgraf}}

\newsavebox\mybox
\newenvironment{aquote}[1]
  {\savebox\mybox{#1}\begin{quote}}
  {\signed{\usebox\mybox}\end{quote}}

%disabilita colore link
\hypersetup{%
    pdfborder = {0 0 0}
}

\begin{document}

\noindent Andrea Gadotti \hfill 22/11/2014

\paragraph{Exercise 27.} First, we want show that
\[
m_k := \E[X^k] = \frac{1}{\sqrt{2\pi}} \int_\R x^k e^{-\frac{x^2}{2}} dx =
\begin{cases}
(k-1)!! & \text{if $k$ is even;}\\
0 & \text{if $k$ is odd.}
\end{cases}
\]
Let's define
\[
I(k):= \int_\R x^k e^{-\frac{x^2}{2}} dx.
\]
If we can prove the following claim, we are done.
\begin{claim}{}
For all $p \in \N$, we have 
\begin{enumerate}
\item $I(2p)=(2p-1)!!\sqrt{2\pi}$;
\item $I(2p+1)=0$. 
\end{enumerate}
\begin{claimproof} Both the proofs proceed by induction on $p$.
\begin{enumerate}
\item Suppose $p=0$. Then 
\[
I(2p)=I(0)= \int_\R e^{-\frac{x^2}{2}} dx = \sqrt{2\pi} = (2\cdot 0 -1)!!\sqrt{2\pi}
\]
Suppose now $p>0$. We have
\begin{align*}
I(2p) 
&= \int_\R x^{2p} e^{-\frac{x^2}{2}} dx\\
&= \int_\R -x^{2p-1} \left[ -x e^{-\frac{x^2}{2}} \right] dx\\
&= -x^{2p-1} e^{-\frac{x^2}{2}} \Bigg|_{-\infty}^{+\infty} + (2p-1) \int_\R x^{2p-2} e^{-\frac{x^2}{2}} dx\\
&=0 + (2p-1) [(2p-3)!! \sqrt{2\pi}]\\
&= (2p-1)!!\sqrt{2\pi}.
\end{align*}
\item For $p=0$ we obtain
\[
I(2p+1)=I(1)= \int_\R x e^{-\frac{x^2}{2}} dx = \E[X] = 0.
\]
For $p>0$, proceeding similarly to point (1), we obtain
\[
I(2p+1)= \int_\R x^{2p+1} e^{-\frac{x^2}{2}} dx = -x^{2p} e^{-\frac{x^2}{2}} \Bigg|_{-\infty}^{+\infty} + 2p \int_\R x^{2p-1} e^{-\frac{x^2}{2}} dx = 0+0=0.
\]
\end{enumerate}
\end{claimproof}
\end{claim}\ \\
Now we want to show that
\[
|m|_k := \E[|X|^k] = \frac{1}{\sqrt{2\pi}} \int_\R |x|^k e^{-\frac{x^2}{2}} dx =
\begin{cases}
(k-1)!! & \text{if $k$ is even;}\\
(k-1)!!\sqrt{\frac{2}{\pi}} & \text{if $k$ is odd.}
\end{cases}
\]
If $k$ is even, the statement follows trivially by $m_k=|m|_k$. Suppose then $p$ odd. If we can prove the following claim, we are done.
\begin{claim}{}
For all $p \in \N$, we have
\[
\int_\R |x|^{2p+1} e^{-\frac{x^2}{2}} dx = (2p)!! \cdot 2.
\]
\begin{claimproof} The proof proceeds by induction on $p$.
Suppose $p=0$. Then 
\[
\int_\R |x|^{2p+1} e^{-\frac{x^2}{2}} dx = \int_\R |x| e^{-\frac{x^2}{2}} dx = 2 \int_0^{+\infty} |x| e^{-\frac{x^2}{2}} dx = -2 e^{-\frac{x^2}{2}} \Bigg|_0^{+\infty} = 0+2 = (2\cdot 0)!! \cdot 2.
\]
Suppose now $p>0$. We have
\begin{align*}
I(2p+1) 
&= \int_\R |x|^{2p+1} e^{-\frac{x^2}{2}} dx\\
&= \int_\R -x^{2p} \left[ -|x| e^{-\frac{x^2}{2}} \right] dx\\
&= 2 \int_0^{+\infty} -x^{2p} \left[ -x e^{-\frac{x^2}{2}} \right] dx\\
&= -2x^{2p} e^{-\frac{x^2}{2}} \Bigg|_0^{+\infty} + 4p \int_0^{+\infty} x^{2p-1} e^{-\frac{x^2}{2}} dx\\
&= 0+ 2p \int_\R |x|^{2p-1} e^{-\frac{x^2}{2}} dx\\
&= 2p [(2p-2)!!\cdot 2] = (2p)!!\cdot 2
\end{align*}
\end{claimproof}
\end{claim}\ \\
In order to find the characteristic function, we will soon need the following:
\begin{lemma}
For every $p \in \N_{\geq 1}$ we have
\[
\log 1 \log 2 + ... + \log p \geq \int_1^p \log x dx.
\]
\begin{proof}
\begin{align*}
\log 1 + \log 2 + ... + \log p 
&= 0 + \log 2 \int_1^2 1 dx + \log 3 \int_2^3 1 dx + ... + \log p \int_{p-1}^p 1 dx\\
&= \int_1^2 \log 2 dx + \int_2^3 \log 3 dx + ... + \int_{p-1}^p \log p dx\\
&\geq \int_1^2 \log x dx + \int_2^3 \log x dx + ... + \int_{p-1}^p \log x dx\\
&= \int_1^p \log x dx,
\end{align*}
where the inequality holds since $\log$ is an increasing function, thus $\log x \leq \log (i+1)$ for all $x \in [i,i+1]$.
\end{proof}
\end{lemma}
To lighten the notation, suppose WLOG that $k=2p$ for some $p \in \N$. Observe now that
\begin{align*}
\lim_{k \to +\infty} \left( \frac{|m|_k}{k!} \right)^{\frac{1}{k}}
&\leq \lim_{k \to +\infty} \left( \frac{(k-1)!!}{k!} \right)^{\frac{1}{k}}\\ 
&= \lim_{k \to +\infty} \left( \frac{1}{k!!} \right)^{\frac{1}{k}} \tag{since $k!! \geq (k/2)!$}\\
&\leq \lim_{k \to +\infty} \left( \frac{1}{(k/2)!} \right)^{\frac{1}{k}}\\
&= \lim_{p \to +\infty} \left( \frac{1}{p!} \right)^{\frac{2}{p}}\\
&= \lim_{p \to +\infty} e^{ {\frac{2}{p}} \log \left( \frac{1}{p!} \right)}\\
&= \lim_{p \to +\infty} e^{ - {\frac{2}{p}} \left( \log p! \right)}\\
&= \lim_{p \to +\infty} e^{ - {\frac{2}{p}} \left( \log 1 + \log 2 + ... + \log p \right)}\\
&\leq \lim_{p \to +\infty} e^{ - {\frac{2}{p}} \int_1^p \log x dx} \tag{thanks to the lemma}\\
&= \lim_{p \to +\infty} e^{ - {\frac{2}{p}} \left( p \log p -p +1 \right) }\\
&= \lim_{p \to +\infty} e^{ - 2 \log p +2 -\frac{2}{p} } = e^{-\infty} = 0.
\end{align*}
This means that we can apply the corollary we saw during the lecture, i.e.
\[
\phi(t)= \sum_{k=0}^\infty m_k \frac{(it)^k}{k!}.
\]
Since $m_k=0$ if $k$ is odd, we obtain
\[
\sum_{k=0}^\infty m_k \frac{(it)^k}{k!} 
= \sum_{h=0}^\infty m_{2h} \frac{(it)^{2h}}{(2h)!} 
= \sum_{h=0}^\infty (2h-1)!! (-1)^h \frac{t^{2h}}{(2h)!}
= \sum_{h=0}^\infty (-1)^h \frac{t^{2h}}{(2h)!!}.
\]
It is very easy to check that $(2h)!!=2^h h!$, for all $h \in \N$. Therefore
\[
\sum_{h=0}^\infty (-1)^h \frac{t^{2h}}{(2h)!!} 
= \sum_{h=0}^\infty (-1)^h \frac{t^{2h}}{2^h h!}
= \sum_{h=0}^\infty \frac{\left( -\frac{t^2}{2} \right)^h}{h!}
= e^{-\frac{t^2}{2}}.
\]
That is, $\phi(t)= e^{-\frac{t^2}{2}}$.

\paragraph{Exercise 28.}
\[
\phi_{aX+b}(t) = \E \left[ e^{it(aX+b)} \right] = \E \left[ e^{itaX}e^{itb} \right] = \E \left[ e^{itaX} \right] \E \left[ e^{itb} \right] = \phi_X(at) e^{itb}.
\]
Thus, since $X \sim N(\mu, \sigma^2) \imp X=\sigma Y+\mu$ with $Y \sim N(0,1)$, we have
\[
\phi_X(t)=\phi_{\sigma Y+\mu}(t)=e^{-\frac{\sigma^2t^2}{2}} e^{it\mu} = e^{it\mu-\frac{1}{2} \sigma^2t^2}.
\]

\paragraph{Exercise 29.} We easily have
\[
\phi_{-X}(t) = \E \left[ e^{i(-t)X} \right] = \phi_X(-t)
\]
and thus
\[
\phi_{X-Y}(t) = \E \left[ e^{it(X+(-Y))} \right] = \phi_X(t) \phi_X(-t).
\]
Now:
\begin{itemize}
\item[a)] Consider $f(t):=e^{2(\cos t -1)}$. We want to find explicity a random variable $Z$ such that $\phi_Z(t)=f(t)$. Let $Y$ be a Bernoulli r.v. with $\P[0]=\P[1]=\frac{1}{2}$. Then
\[
\phi_Y(t)=\frac{1}{2} e^{-it} + \frac{1}{2} e^{it} = \cos t.
\]
Now let $N$ be distributed like a Poisson distribution of parameter $\lambda$, and let $(Y_j)$ be i.i.d. random variables, independent with $N$. Define 
\[
S=\sum_{j=1}^N Y_j.
\]
We have 
\begin{align*}
\phi_S(t)= \E \left[ e^{itS} \right] 
&= \E \left[ e^{it \sum\limits_{j=1}^N Y_j} \right] \\
&= \E \left[ \sum_{n \geq 0} e^{it \sum\limits_{j=1}^n Y_j} \1_{\{N=n\}} \right] \\
&= \sum_{n \geq 0} \E \left[ e^{it \sum\limits_{j=1}^n Y_j} \right] \cdot \E \left[ \1_{\{N=n\}} \right] \tag{$N$ and $(Y_j)$ indep.} \\
&= \sum_{n \geq 0} \E \left[ e^{it \sum\limits_{j=1}^n Y_j} \right] \cdot \P[N=n] \\
&= \sum_{n \geq 0} \underbrace{\phi_{Y_1}(t)^n}_{\mathbb{E}e^{it \sum_{j=1}^n Y_j}} \cdot \underbrace{e^{-\lambda} \frac{\lambda^n}{n!}}_{\P[N=n]}= e^{\lambda(\phi_{Y_1}(t)-1)}.
\end{align*}
Therefore, if we take
\[
Z:= \sum_{j=1}^N Y_j
\]
with $(Y_j)$ i.i.d. Bernoulli symmetric r.v. with range $\{1,-1\}$ and $N \sim \text{Pois}(2)$ independent with $(Y_j)$, we obtain 
\[
\phi_Z(t) = e^{2(\cos t -1)},
\]
as wanted.
\item Consider $f(t)=e^{-|t|^3}$. We have
\[
f'(t)=-3|t|te^{-|t|^3}
\]
and
\[
f''(t)= -3 \left( |t|e^{-|t|^3} + |t|e^{-|t|^3} -3|t|te^{-|t|^3}|t|t \right).
\]
So $f'(0)=f''(0)=0$. Let $X$ be a random variable. We now that $\phi'_X(0)=i\E[X]$. So, if $f(t)=\phi_X(t)$, we would have $\E[X]=0$. But since $\phi''_X(0)=-\E[X^2]=\E[(X-0)^2]=\Var[X]$, we would have $\Var[X]=0$. Therefore $X$ must be the constant function in $0$. But the characteristic function of $X \equiv 0$ is trivially $e^{it0}=1 \neq e^{-|t|^3}$. Thus $e^{-|t|^3}$ is not the characteristic function of any random variable.
\end{itemize}
\end{document}